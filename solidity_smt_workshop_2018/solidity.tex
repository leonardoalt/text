\section{Solidity}
\label{section:solidity}

Solidity is a programming language specifically developed to write smart contracts
which run on the Ethereum Virtual Machine. It is a statically-typed curly-braces
language with a syntax similar to Java. The main source code elements are
called \emph{contracts} and are similar to classes in other languages.
Contract-level variables in Solidity are persisted in storage while
local variables and function parameters only have a temporary lifetime.
Among others, Solidity has integer data types of various sizes (up to 256 bits,
the word size of the EVM), address types and an associative array
type called \emph{mapping} which can only be used for contract-level variables.

The source code in Fig.~\ref{fig:tokenContract} shows a minimal example of a token contract.
Users are identified by their addresses and initially, all tokens are
owned by the creator of the contract, but
anyone who owns tokens can transfer an arbitrary amount to other addresses.
Authentication is implicit in the fact that the address from which
a function is called can be accessed through the global variable
\texttt{msg.sender}. In practice, this is enforced by checking a
cryptographic signature on the transaction that is sent through the network.

\begin{figure}
\begin{verbatim}
contract Token {
    /// The main balances / accounting mapping.
    mapping(address => uint256) balances;

    /// Create the token contract crediting `msg.sender` with
    /// 10000 tokens.
    constructor() public {
        balances[msg.sender] = 10000;
    }

    /// Transfer `_value` tokens from `msg.sender` to `_to`.
    function transfer(address _to, uint256 _value) public {
        require(balances[msg.sender] >= _value);
        balances[msg.sender] -= _value;
        balances[_to] += _value;
    }
}
\end{verbatim}
\caption{Example of a token contract}
\label{fig:tokenContract}
\end{figure}

The \texttt{require} statement inside the function \texttt{transfer} is used
to check a precondition at run-time: If its argument evaluates to false,
the execution terminates and any previous change to the state is reverted.
Here, it prevents tokens being transferred that are not actually available.

In general, invalid input should be caught via a failing \texttt{require}.
The related \texttt{assert} statement can be used to check postconditions.
The idea behind is that it should never be possible to reach a failing
assert.
\texttt{assert} essentially\footnote{There are two main differences however: As opposed to \texttt{require},
\texttt{assert} will result in all remainder \emph{gas} to be consumed and \texttt{require} allows a custom message to be returned.}
has the same effect as \texttt{require}, but
is encoded differently in the bytecode. Verification tools on bytecode
level (as opposed to the high-level approach described in this article)
typically check whether it is possible to reach an assert in any way.

We now show how an \texttt{assert} can be introduced into the \texttt{transfer} function
to perform a simple invariant check.
\begin{verbatim}
function transfer(address _to, uint256 _value) public {
    require(balances[msg.sender] >= _value);
    uint256 sumBefore = balances[msg.sender] + balances[_to];
    balances[msg.sender] -= _value;
    balances[_to] += _value;
    uint256 sumAfter = balances[msg.sender] + balances[_to];
    assert(sumBefore == sumAfter);
}
\end{verbatim}

The \texttt{assert} checks that the sum of the balances in the two
accounts involved did not change due to the transfer. Currently,
the \texttt{assert} statement is not removed by the compiler, even
if the formal analysis module can prove that it never fails.

Note that in the general case, \verb+balances[_to]+ can overflow
and thus an analysis tool might flag this assert as potentially
failing. In this specific example, though, the amount of available tokens
is too small for this to happen.
